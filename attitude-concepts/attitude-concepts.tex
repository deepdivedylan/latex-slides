\documentclass[aspectratio=169]{beamer}

% because we need to claim weird things
\newtheorem{claim}{Claim}
\newtheorem{defn}{Definition}
%\newtheorem{lemma}{Lemma}
\newtheorem{thm}{Theorem}
\newtheorem{vita}{Vit\ae}
\newtheorem{qotd}{Quote of the Day}

\usepackage{algorithm}
\usepackage{algpseudocode}
\usepackage{listings}
\usepackage{color}
\usepackage{graphics}
\usepackage{ulem}
\bibliographystyle{unsrt}

% background image
\usebackgroundtemplate%
{%
    \includegraphics[width=\paperwidth,height=\paperheight]{../artifacts/stemulus.pdf}%
}
\setbeamertemplate{caption}[numbered]
\lstset{%
	breaklines=true,
	captionpos=b,
	frame=single,
	keepspaces=true
}

% page numbers
\addtobeamertemplate{navigation symbols}{}{%
    \usebeamerfont{footline}%
    \usebeamercolor[fg]{footline}%
    \hspace{1em}%
    \insertframenumber/\inserttotalframenumber
}

% presentation header
\usetheme{Warsaw}
\title{Programming Concepts}
\author{Dylan Lane McDonald}
\institute{CNM STEMulus Center\\Web Development with PHP}
\date{\today}

\begin{document}
\lstset{language=HTML}
\begin{frame}
\titlepage
\end{frame}

\begin{frame}
\frametitle{Outline}
\tableofcontents
\end{frame}

\section{Who Are We?}
\subsection{Who Are We?}
\begin{frame}
\frametitle{Who Are We?}
What are we training to be?
\begin{itemize}
	\item Software Engineer: one who writes software
	\pause
	\item Computer Scientist: one who uses sound mathematical principles to solve problems
	\pause
	\item Software Architect: one who designs software programs so software engineers can implement them
	\pause
	\item Web Developer: one who writes software for the web programming paradigm
\end{itemize}
\pause
So who are we\dots?
\pause
\mbox{}\\
\textbf{All four of them!}
\end{frame}

\subsection{What Are We Doing?}
\begin{frame}
\frametitle{What Are We Doing?}
As mentioned in the last slide, we are learning the web programming paradigm.
\pause
\begin{defn}
A \textbf{paradigm} is how a programming language is structured and thought about. Each paradigm is specialized to the problems it aims to solve.
\end{defn}
\pause
There are five major paradigms and dozens of minor paradigms. Each paradigm is specialized to a class of problems. Problems can be solved by multiple paradigms and there are advantages and disadvantages of each.
\end{frame}

\begin{frame}
\frametitle{Programming Paradigms}
Common paradigms include:
\begin{itemize}
	\item Imperative: programming step-by-step using procedures (C)
	\pause
	\item Functional: programming using mathematical functions (Haskell)
	\pause
	\item Object-oriented: programming by modeling each player in the program (Java)
	\pause
	\item Logic: programming by axiomatic logical statements (Prolog)
	\pause
	\item Symbolic: programming by manipulating formulas \& symbols (LISP)
\end{itemize}
\end{frame}

\section{How Do We Succeed?}
\subsection{How Do We Use It?}
\begin{frame}
\frametitle{How Do We Use It?}
Programming languages are used in three different ways:
\begin{itemize} 
	\item Compiled: Source code is passed to a \textit{compiler}, which takes the source code and transforms in into \textit{machine code}, which is directly executable by the CPU (C)
	\item Interpreted: Source code is passed to an \textit{interpreter}, which executes the source code line-by-line as the code is encountered (PHP)
	\item Hybrid: Source code is compiled into an intermediate, optimized format called \textit{byte code} that is saved and passed to an interpreter when it needs to execute (Java)
\end{itemize}
The other languages we will be covering, JavaScript \& SQL, are also interpreted.
\end{frame}

\subsection{How Do We Write Better Code?}
\begin{frame}
\frametitle{How Do We Write Better Code?}
Code maintainability is essential. Non readable, non maintainable code precludes development and is a hindrance to the team. Code can be made easy-to-read by:
\begin{itemize}
	\item Properly indenting the code at each logical level
	\item Using variable names that concisely describe what the variable does
	\item Sticking to one variable capitalization convention\footnote{In this class, ``camelBackVariables'' are preferred.}
	\item Thoroughly commenting code as you write it
	\item Exercising good \texttt{git} etiquette\dots
	\pause
	\begin{itemize}
		\item Commit early, commit often!
		\item Write descriptive comments in the \texttt{git commit} journal
	\end{itemize}
\end{itemize}
\end{frame}

\subsection{How Should I Act?}
\begin{frame}
\frametitle{How Should I Act?}
Attitude is everything! Computer Science is a challenging field in which you are part of a team. Separate yourself by:
\begin{itemize}
	\item \textbf{\underline{\emph{TENACITY}}}: Things \textbf{will} go wrong at some point. Great programmers persist, keep their frustrations in check, and do what it takes to climb the inevitable brick walls.
	\item \textbf{Teamwork}: Cooperation and being approachable is a must. You will be working side-by-side with people and the more you help the team, the more you help yourself.
	\item \textbf{Empathy}: Again, this can be a frustrating field at times. Understanding and taking the point of view of an upset end-user will not only defuse a volatile situation, but also help you arrive at a solution.
\end{itemize}
\end{frame}

\end{document}