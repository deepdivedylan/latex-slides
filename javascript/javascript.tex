\documentclass[aspectratio=169]{beamer}

% because we need to claim weird things
\newtheorem{claim}{Claim}
\newtheorem{defn}{Definition}
%\newtheorem{lemma}{Lemma}
\newtheorem{thm}{Theorem}
\newtheorem{vita}{Vit\ae}
\newtheorem{qotd}{Quote of the Day}

\usepackage{algorithm}
\usepackage{algpseudocode}
\usepackage{listings}
\usepackage{color}
\usepackage{graphics}
\usepackage{ulem}
\bibliographystyle{unsrt}

% background image
\usebackgroundtemplate%
{%
    \includegraphics[width=\paperwidth,height=\paperheight]{../artifacts/stemulus.pdf}%
}
\setbeamertemplate{caption}[numbered]
\lstset{%
	breaklines=true,
	captionpos=b,
	frame=single,
	keepspaces=true
}

% page numbers
\addtobeamertemplate{navigation symbols}{}{%
    \usebeamerfont{footline}%
    \usebeamercolor[fg]{footline}%
    \hspace{1em}%
    \insertframenumber/\inserttotalframenumber
}

% presentation header
\usetheme{Warsaw}
\title{Introduction to JavaScript}
\author{Dylan Lane McDonald}
\institute{CNM STEMulus Center\\Web Development with PHP}
\date{\today}

\begin{document}
\lstset{language=HTML}
\begin{frame}
\titlepage
\end{frame}

\begin{frame}
\frametitle{Outline}
\tableofcontents
\end{frame}

\section{About JavaScript}
\subsection{History of JavaScript}
\begin{frame}
\frametitle{History of JavaScript}
JavaScript began its life at Netscape in 1995. Originally slated for Netscape 2.0, the language was originally called \textit{LiveScript} and subsequently renamed to \textit{JavaScript} shortly before Netscape 2.0's release. This begs the question: is JavaScript similar to Java?

\mbox{}\\
\pause
\textbf{ABSOLUTELY NOT!!!!}

\mbox{}\\
\pause
JavaScript's similarities to Java are purely superficial. The syntax is derived from a common ancestor, C, and both use C-like syntax. The similarities stop there.
\end{frame}

\subsection{Features of JavaScript}
\begin{frame}
\frametitle{Features of JavaScript}
JavaScript is an \textbf{interpreted} language run by a JavaScript engine, normally a component of the end user's web browser. Some of the key features of JavaScript are:
\begin{itemize}
	\item \textbf{Imperative}: Functions can be written to perform specific tasks
	\item \textbf{Object Oriented}: Objects can be modeled around what the system is
	\item \textbf{Dynamically Typed}: The variable type (e.g., String, Integer, \dots) are decided at run time 
\end{itemize}
JavaScript is enabled on most end user's web browsers and is used in creating responsive, dynamic, and fun sites.
\end{frame}

\section{Events \& Closures}
\subsection{Events}
\begin{frame}
\frametitle{Javascript Events}
An event can roughly be categorized into one or both of the following categories:
\begin{itemize}
	\item Something the browser does
	\item Something the user does
\end{itemize}

\pause
\mbox{}\\
Events are the centerpiece of creating interactive web sites. Whether flat JavaScript or a framework such as Angular or jQuery are used, reacting to events is the key to creating dynamic and interactive web sites.

\pause
\mbox{}\\
Events map everything from user interaction such as key presses, mouse movements, and scrolls to network events such as loads and going offline.
\end{frame}

\begin{frame}
\frametitle{Example Events}
Events are a reaction to a user's action on a web site. Table \ref{tbl:events} lists a very small subset of possible JavaScript events.
\begin{table}
\begin{tabular}{|l|l|}
\hline
\textbf{Event} & \textbf{Comment}\\
\hline
click & when a user clicks on an element\\
\hline
change & when a user changes an input field\\
\hline
drag & when a user drags an element\\
\hline
drop & when a user drops an element onto another\\
\hline
scroll & when a user scrolls (desktop) or swipes (mobile)\\
\hline
\end{tabular}
\caption{Common JavaScript Events}
\label{tbl:events}
\end{table}

As always, a more exhaustive list of all the possible events are available at the Mozilla Developer Network. \cite{mdn} 
\end{frame}

\subsection{Closures}
\begin{frame}
\frametitle{Closures}
\begin{defn}
A \textbf{closure} is a JavaScript function that generates another function. The advantage of a closure is to generalize functionality for use cases that only vary slightly.
\end{defn}

\pause
An example use case of a closure is to greet the user in many different languages. Here, we have a constant and a variable:
\begin{itemize}
	\item \textbf{Constant:} the user's name
	\item \textbf{Variable:} the exact words in the user's native language
\end{itemize}

\pause
Using a closure, one can boilerplate the words in the particular language and concentrate on the only variable: the user's name. The use of closures is vast in JavaScript frameworks such as Angular and jQuery, where closures are applied to multiple elements at once using class selectors.
\end{frame}

\begin{frame}
\frametitle{Works Cited}
\bibliography{javascript}
\end{frame}

\end{document}